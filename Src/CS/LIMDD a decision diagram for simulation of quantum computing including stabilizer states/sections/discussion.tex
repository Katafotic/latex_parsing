\section{Discussion \label{sec:discussion}}

We have introduced \limdd, a novel decision diagram-based method to simulate quantum circuits, which enables polynomial-size representation of a strict superset of stabilizer states and of the states represented by polynomially-large \qmdds.
To prove this, we have shown the first lower bounds on the size of \qmdds for stabilizer states:
they are exponential-size for certain families of stabilizer states.
\limdds achieve a more succinct representation by representing states up to 
local invertible maps which uses single qubit (local) operations from a group $G$.
We have investigated the choices $G=\pauli$, $G=\braket{Z}$ and $G=\braket{X}$,
and found that any choice suffices for an exponential advantage over \qmdds;
notably, the choice $G=\pauli$ allows us to succinctly represent stabilizer states.
We also defined reduction rules for Pauli-\limdd and showed that for each set of 
quantum states which are equivalent under local Pauli operations,
modulo normalization factor, has a unique reduced Pauli-\limdd as representative.

Furthermore, we showed how to simulate arbitrary quantum circuits, encoded as Pauli-\limdds.
The resulting algorithms are often faster than for \qmdds.
In contrast to \qmdds, Clifford circuits (initialized to $\ket{0}$)
can be simulated by Pauli-\limdds in polynomial time.
This in itself is not an interesting feat, since efficient simulation methods for
the stabilizer regime have been known since the Gottesman-Knill theorem.
However, we also showed that Pauli-\limdds can efficiently simulate
a circuit family we call Hamming weight-controlled Clifford circuits.
And we provide empirical evidence that these circuits are hard for stabilizer-rank
based simulation methods.
%we were strengthened in this exception by exploratory numerics.

An obvious next step is to investigate other choices for $G$.
Of interest are both the representational capabilities of such diagrams
(do they represent interesting states?), and the algorithmic capabilities
(can we still find efficient algorithms which make use of these diagrams?).
In this vein, an important question is what the relationship is between \glimdds
(for various choices of $G$) and existing formalisms for the classical simulation of quantum circuits, such as those based on match-gates~\cite{terhal2001classical,jozsa2008matchgates,hebenstreit2020computational} and tensor networks~\cite{orus2014practical}.
It would also be interesting to compare \limdds to graphical depictions of quantum computation, 
following similar work for \qmdds~\cite{vilmart2021quantum}.
We leave empirical evaluations into the consequences of \limdd methods for efficient
quantum circuit simulation, for future work. This would obviously include a 
comparison with stabilizer rank simulation~\cite{bravyi2019simulation}.

Finally, we note that the current definition of \limdd imposes a strict total order
over the qubits along every path from root to leaf.
It is known that the chosen order can greatly influence the size of the DD~\cite{rudell1993dynamic,wegener2000branching},
making it interesting to investigate variants of \limdds with a flexible ordering.

