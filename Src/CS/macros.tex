
\usepackage{xspace}

\usetikzlibrary{intersections,fit,shapes.misc, decorations.markings}

\newcommand\marktopleft[1]{%
    \tikz[overlay,remember picture] 
        \node (marker-#1-a) at (-.5ex,1.5ex) {};%
}
\newcommand\markbottomright[1]{%
    \tikz[overlay,remember picture] 
        \node (marker-#1-b) at (-.5ex,-0.1ex) {};%
    \tikz[overlay,remember picture,inner sep=3pt]
        \node[draw=none,fill=red!20,rectangle,fill opacity=.2,fit=(marker-#1-a.center) (marker-#1-b.center)] {};%
}

%\newenvironment{rcases}
%  {\left.\begin{aligned}}
%  {\end{aligned}\right\rbrace}

\newcommand\marktopleftb[1]{%
    \tikz[overlay,remember picture] 
        \node (marker-#1-a) at (-.5ex,1.5ex) {};%
}
\newcommand\markbottomrightb[1]{%
    \tikz[overlay,remember picture] 
        \node (marker-#1-b) at (-.5ex,-0.1ex) {};%
    \tikz[overlay,remember picture,inner sep=3pt]
        \node[draw=none,fill=blue!20,rectangle,fill opacity=.2,fit=(marker-#1-a.center) (marker-#1-b.center)] {};%
}

\def\tikzmark#1{\tikz[remember picture,overlay]\node[inner ysep=0pt,anchor=base](#1){\strut};}


\makeatletter
\def\namedlabel#1#2{\begingroup
    #2%
    \def\@currentlabel{#2}%
    \phantomsection\label{#1}\endgroup
}
\makeatother

%\theoremstyle{definition}
%\newtheorem{definition}{Definition}[section]

%\setlength{\parindent}{0em}
%\setlength{\parskip}{.2em}

%\captionsetup[figure]{labelfont={bf,it},textfont={it}}

%\setcounter{secnumdepth}{3}

\renewcommand\sectionautorefname{Section}
\renewcommand\subsectionautorefname{Section}
\renewcommand\subsubsectionautorefname{Section}
%\renewcommand\subsubsectionautorefname{\textsection}
%\renewcommand\corollaryautorefname{Cor.}
%\renewcommand\figureautorefname{Fig.}
%\renewcommand\exampleautorefname{Ex.}
\newcommand\algorithmautorefname{Algorithm}
\renewcommand\appendixautorefname{Appendix}
%\renewcommand\algautorefname{Alg.}
%\renewcommand\observationautorefname{Obs.}
\renewcommand\itemautorefname{}
%\renewcommand\lemmaautorefname{Lemma}
%\renewcommand\theoremautorefname{Th.}
\renewcommand\equationautorefname{Eq.}
%%\newcommand{\subfigureautorefname}{\figureautorefname}
%\providecommand*\definitionautorefname{Def.}

\makeatletter
\patchcmd{\ALG@step}{\addtocounter{ALG@line}{1}}{\refstepcounter{ALG@line}}{}{}
\newcommand{\ALG@lineautorefname}{Line}
\makeatother

\newcommand\obj[1]{\textbf{Obj.~#1}}
%\newcommand\problem[1]{\textbf{Problem~#1}}
\newcommand\task[1]{\textbf{Task~#1}}


\newcommand{\cbox}[2][yellow]{%
  \colorbox{#1}{\parbox{\dimexpr\linewidth-2\fboxsep}{\strut #2\strut}}%
}

%\definecolor{emph}{rgb}{1,0.5,0}
%\renewcommand\emph[1]{{\color{emph}#1}}

\renewcommand\phi{\varphi}


\newcommand\defaccr[2]{\newcommand#1{#2\xspace}}
\newcommand\defmath[2]{\newcommand#1{\ensuremath{#2}\xspace}}
\newcommand\concept[1]{\textit{#1}}


\defmath{\img}{\mathtt{image}}
\defmath{\pre}{\mathtt{preimage}}
\defmath{\apre}{\mathtt{\forall preimage}}

\defmath{\Post}{\mathit{Postfix}}

\newcommand\ccode[1]{\texttt{#1}} 


\newcommand{\overarrowi}[1]{\xrightarrow{#1}}
\newcommand{\overarrow}[1]{
  \mathchoice{\raisebox{-3pt}{ $\overarrowi{#1}$ }}
             {\raisebox{-3pt}{ $\overarrowi{#1}$ }}
             {\raisebox{-3pt}{ $\overarrowi{#1}$ }}
             {\raisebox{-3pt}{ $\overarrowi{#1}$ }}}

%containers
\let\set\undefined

\providecommand{\tuple}[1]{\ensuremath{\left( #1 \right)}}
\providecommand{\set}[1]{\ensuremath{\left\lbrace #1 \right\rbrace}}
\providecommand{\sequence}[1]{\ensuremath{\left( #1 \right)}}
\providecommand{\sizeof}[1]{\ensuremath{\left\vert{#1}\right\vert}}
\providecommand{\vect}[1]{\ensuremath{( \begin{matrix} #1 \end{matrix} )}}
\providecommand{\always}[1]{\ensuremath{\left[ #1 \right]}}
%\newcommand{\powerset}[1]{\wp({#1})}
\newcommand{\powerset}[1]{\ensuremath{\mathbf{2}^{#1}}}
% Semantics
\newcommand{\eval}[2][]{\ensuremath{\llbracket #2\rrbracket^{#1}}}


\providecommand{\gen}[1]{\ensuremath{\left\langle #1 \right\rangle}}



\newcommand{\id}[1][]{\ensuremath{\mathbb{I}_{#1}}\xspace}
\defmath{\bool}{\ensuremath{\mathbb{B}}}
\defmath{\nat}{\ensuremath{\mathbb{N}}}
\defmath{\complex}{\ensuremath{\mathbb{C}}}
\defmath{\real}{\ensuremath{\mathbb{R}}}
\defmath{\integers}{\ensuremath{\mathbb{Z}}}
\defmath{\conditionalind}{\mathrel{\text{\scalebox{1.07}{$\perp\mkern-10mu\perp$}}}}
\defmath{\dx}{\partial x}
\defmath{\ddx}{\sfrac{\partial}{\partial x}}
\defmath{\half}{\textstyle{\frac{1}{2}}}

\newcommand{\bracket}[1]{\left(#1\right)}
\newcommand{\prob}[1]{P\left[#1\right]}
\newcommand{\underbraceset}[2]{\underset{#1}{\underbrace{#2}}}
\newcommand{\rfrac}[2]{\ ^{#1} \!/_{#2}}
\newcommand{\atmost}[1]{\ensuremath{\mathcal{O}(#1)}}
\newcommand{\ceil}[1]{\left\lceil #1 \right\rceil}
\newcommand{\floor}[1]{\left\lfloor #1 \right\rfloor}


\defmath\Exists{\mathit{Exists}}
\defmath\PlusExists{\mathit{PlusExists}}
\defmath\var{\mathit{var}}
\defmath\calciso{\mathsf{calciso}}



% Abs, Floor, Ceil
\providecommand{\abs}[1]{\lvert#1\rvert}
%\providecommand{\floor}[1]{\lfloor#1\rfloor}
%\providecommand{\ceil}[1]{\lceil#1\rceil}

\newcommand{\defn}{\,\triangleq\,}

% Rotate text
\newcommand{\turner}[3][10em]{% \turn[<width>]{<angle>}{<stuff>}
  \rlap{\rotatebox{#2}{\begin{varwidth}[t]{#1}#3\end{varwidth}}}%
}



\tikzstyle{oval} = [state, ellipse, minimum size=4mm, inner sep=0.5mm, node distance=1cm]
\tikzset{every picture/.style={->,thick}}

% For BDDs:
\tikzstyle{leaf}=[draw, rectangle,minimum size=5mm, inner sep=3pt]
\tikzstyle{var}=[circle,draw=black!70,solid,thick,minimum size=6mm]
\tikzstyle{bdd}=[regular polygon, regular polygon sides=3, draw=black!70,solid,thick,inner sep=0.5mm]
\tikzstyle{n}=[->,loosely dashed,thick]
\tikzstyle{p}=[->,solid,thick]
\tikzstyle{b}=[->,densely dashdotted,ultra thick]




\defmath\before{\prec}
\defmath\beforeq{\preccurlyeq}




%\usepackage{physics}
% package physics clashes with another package; now manually adding the
% commands needed:
\newcommand{\expval}[1]{\langle #1 \rangle}
\newcommand{\dyad}[1]{| #1 \rangle \langle #1 |}
\newenvironment{smallmat}{\left[\begin{smallmatrix}}{\end{smallmatrix}\right]}
\newcommand{\Iso}{\ensuremath{\text{Iso}}}
\newcommand{\Stab}{\ensuremath{\text{Stab}}}
\newcommand{\Aut}{\ensuremath{\text{Stab}}}

\newcommand\local[5]{\begin{smallmat}#5{#1} & #5{#2} \\ #5{#3} & #5{#4}\end{smallmat}\xspace}
\newcommand\anti[1]{\begin{smallmat}0 & #1 \\ 1 & 0\end{smallmat}\xspace}
\newcommand\diag[1]{\begin{smallmat}1 & 0 \\  0 & #1\end{smallmat}\xspace}

\defaccr{\bdd}{\textsf{BDD}}
\defaccr{\bdds}{\textsf{BDD}s}
\defaccr{\qmdd}{\textsf{QMDD}}
\defaccr{\add}{\textsf{ADD}}
\defaccr{\isoqmdd}{\textsf{LIMDD}}
\defaccr{\limdd}{\textsf{LIMDD}}
\defaccr{\qmdds}{\textsf{QMDD}s}
\defaccr{\adds}{\textsf{ADD}s}
\defaccr{\isoqmdds}{\textsf{LIMDD}s}
\defaccr{\limdds}{\textsf{LIMDD}s}
\defaccr{\glimdd}{\ensuremath{G}-\limdd}
\defaccr{\glimdds}{\ensuremath{G}-\limdds}

\newcommand{\tim}[1]{\textcolor{blue}{[\textbf{Tim: }#1]}}
\newcommand{\lieuwe}[1]{\textcolor{blue}{\textbf{Lieuwe: }#1}}
\newcommand{\codecomment}[1]{{\small \textcolor{blue}{$\triangleright$ #1}}}

\renewcommand\index{\textsf{idx}\xspace}
%\newcommand\iso{\textsf{iso}}
\newcommand\leaf{\textsf{Leaf}\xspace}
\newcommand\lbl{\textsf{label}\xspace}
\newcommand\highlabel{\textsf{HighLabel}\xspace}
\newcommand\rootlabel{\textsf{RootLabel}\xspace}
\newcommand\unique{\textsc{Unique}\xspace}
\newcommand\cache{\textsc{Cache}\xspace}
\newcommand\autocache{\textsc{StabCache}\xspace}
\newcommand\isocache{\textsc{IsoCache}\xspace}
\newcommand\Edge{\textsc{Edge}\xspace}
\newcommand\Node{\textsc{Node}\xspace}
\newcommand\Pauli{\textsc{Pauli}\xspace}
\newcommand\pauli{\Pauli}
\newcommand\makeedge{\textsc{MakeEdge}\xspace}
\newcommand{\rootedge}{\textsc{RootEdge}\xspace}
%\newcommand{\gmax}{\text{gmax}} % todo Please define the \gmax command -LV

\defmath\oh{\mathcal O}

\defmath\rootlim{B_{\textnormal{root}}}
\defmath\lowlim{B_{\textnormal{low}}}
\defmath\highlim{B_{\textnormal{high}}}
%\defmath\gmax{\textnormal{gmax}}
%\def\gmax{g^{\textnormal{max}}}
\defmath\gmax{g}
\defmath\kmax{\kappa^{\textnormal{final}}}

\newcommand{\alfons}[1]{{\sethlcolor{yellow} \hl{#1}}}
%\newcommand\alfons[1]{\colorbox{blue!30}{#1}}

\defmath\cast{\mathbb C^\ast}

\newcommand\numberthis{\addtocounter{equation}{1}\tag{\theequation}}

%\defmath\bool{\mathcal B}

\DeclareMathOperator*{\argmax}{arg\,max}
\DeclareMathOperator*{\argmin}{arg\,min}

\newcommand\follow[2]{\ensuremath{\textsc{follow}_{#1}(#2)}}
%\defmath\plus{\,\,\raisebox{-.1mm}{\rotatebox{0}{$+\hspace{-1.5mm}\triangleright$}}\,\,}
%\defmath\plus{\,\,\raisebox{-.1mm}{\rotatebox{0}{$+\hspace{-.5mm}\rangle$}}\,\,}
%\defmath\plus{\,\,\raisebox{-.4mm}{\rotatebox{90}{$\pm$}}\,\,}
\defmath\plus{+}
%\newcommand\ledge[2]{\ensuremath{#1.#2}}


\newlength{\pgfcalcparm}
\newlength{\pgfcalcparmm}


\DeclareRobustCommand{\ledge}[3][]{%
  \pgftext{\settowidth{\global\pgfcalcparm}{\scriptsize $\,\,#2\,\,$}}%
  \raisebox{-.8mm}{%
  \tikz{%
    \node[inner sep=0pt] (x){$#1\,\,$};%
    \node[state,inner sep=0pt,minimum size=10pt,right=\pgfcalcparm of x](v){\scriptsize $#3$};%
    \draw (x) to node[above,pos=.5]{\scriptsize $\,#2\,\,$} (v);%
  }%
  }%
}

\DeclareRobustCommand{\lnode}[5][]{%
    \pgftext{\settowidth{\global\pgfcalcparm}{\scriptsize $\,\,#2\,\,$}}%
    \pgftext{\settowidth{\global\pgfcalcparmm}{\scriptsize $\,\,#4\,\,$}}%
  \raisebox{-1.5mm}{%
  \tikz{%
    \node[state,inner sep=0pt,minimum size=10pt] (v){\scriptsize $#1$};%
    \node[state,inner sep=0pt,minimum size=10pt,left=\pgfcalcparm of v](v0){\scriptsize $#3$};%
    \draw[dotted] (v) to node[above,pos=.45]{\scriptsize $#2$} (v0);%
    \node[state,inner sep=0pt,minimum size=10pt,right=\pgfcalcparmm of v](v1){\scriptsize $#5$};%
    \draw (v) to node[above,pos=.45]{\scriptsize $#4$} (v1);%
  }%
  }%
}
%\renewcommand\lnode[5][]{\ensuremath{\ledge{#2}{#3} \plus \ledge{#4}{#5}}}


\newcommand\low[1]{\ensuremath{\textsf{low}(#1)}}
\newcommand\high[1]{\ensuremath{\textsf{high}(#1)}}
\defmath\Low{\ensuremath{\textsf{low}}}
\defmath\High{\ensuremath{\textsf{high}}}
\defmath\LIM{\textsf{LIM}}

\def\findisomorphism{\textsf{FindIsomorphism}\xspace}
\def\getautomorphisms{{\tt GetStabilizerGenSet}\xspace}
\def\isomorphismset{{\tt IsomorphismSet}\xspace}
\def\getsingleisomorphism{\textnormal{{\tt GetIsomorphism}}\xspace}
\def\findautomorphismsetintersection{{\tt IntersectAutomorphismGenSets}\xspace}
\def\findisomorphisminintersection{{\tt FindIsomorphismInIntersection}\xspace}
\def\findisomorphismsetintersection{{\tt IntersectIsomorphismSets}\xspace}

\def\findpauliisomorphism{{\sc FindSingleIsomorphism}}
\def\getpauliautomorphismgenerators{{\sc get-pauli-automorphism-generators}}
\def\automorphismgenerators{{\tt AutomorphismGenSet}}
\def\none{{\tt None}}
\def\horizontalline{\noindent\rule{\textwidth}{1pt} }
\def\findelementincosetintersection{\textsc{FindElementInCosetIntersection}}
\def\findsolutiontoisomorphismequations{\textsc{FindSolutionToIsomorphismEquations}}
\def\psim{\simeq_{\text{Pauli}}}


\defmath\yy{\begin{smallmat}
    0 & y^*\\
    y & 0\\
\end{smallmat}}

\defmath\ww{\begin{smallmat}
      0 & y   \\
      y^* & 0  \\
  \end{smallmat}
}


\def\paulilim{\textnormal{\sc PauliLIM}}
