\begin{table}
\centering
 \makebox[\linewidth]{
    \begin{tabular}{| l | c | c | c | c | c |}
    \hline
&
%representation size
Stabilizer-
&
% hadamard
Apply Hadamard
&
% measuring qubit k
$Z$-measure
&
% CNOT+P+T circuit
Apply $a$ CNOT/$S$ gates
\\
&
%representation size
state-size
&
% hadamard
gate
&
% measuring qubit k
arbitr. qubit
&
% CNOT+P+T circuit
and $t$ T gates, then $X$-measure 1st
\\\hline
        QMDD\cite{zulehner2018advanced}
&
%representation size
$\Omega(2^n)$
&
% hadamard
$O(2^n)$
&
% measuring qubit k
$O(2^n){}^{\ast}$
&
% CNOT+P+T circuit
        $O((a+t)\cdot 2^n) + O(2^n)$
\\\hline
        Extended stabilizer formalism [ref?]
&
%representation size
$\Theta(n^2)$
&
% hadamard
$\Theta(n)$
&
% measuring qubit k
$\Theta(n^2)$
&
% CNOT+P+T circuit
        $O(c^t)$ \textcolor{red}{TODO: fill in $c$}
\\\hline
        unitary-DT-LIM-QMDD (this work)
&
%representation size
$\Theta(n^2)$
&
% hadamard
$O(2^n)$ or $O(n^4)^{\ast\ast}$
&
% measuring qubit k
$O(n^2){}^{\ast}$
&
% CNOT+P+T circuit
$O((a+t)\cdot n) + O(2^n)$
\\\hline
    \end{tabular}
}
	\caption{
		\label{table:runtimes}
		Worst-case time complexity of representing a stabilizer state on $n$ qubits and applying circuits to it for various formalisms.
        The LIM group for the LIM-QMDD is the unitary part of the Dihedral Torus.
        We use the notation $B$-measurement to denote computing $\Pr(\textnormal{outcome}=1)$ for single-qubit measurements in the basis of operator $B$.
${}^{\ast}$Assuming the (LIM-)QMDD has not been normalized yet.
${}^{\ast\ast}$ The polynomial runtime is achieved by converting the LIM-QMDD to the stabilizer formalism, performing the Hadamard gate, and converting back.
}
\end{table}
