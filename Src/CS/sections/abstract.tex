\begin{abstract}
\begin{comment}
	Decision diagrams (DDs) can represent quantum states and circuits, and have proven a powerful method for simulation and synthesis.
	We show that the class of stabilizer states can yield exponentially-sized DDs, despite their well-known tractability and importance.
	To remedy this, we introduce a new DD structure, called Local Invertible Map-DD (\limdd), and provide manipulation operations for  supporting simulation and synthesis.
	We prove that the set of poly-sized \limdd states strictly contains the union of stabilizer states and existing DD-based methods.
	% use this iff we have zip states
	%We also provide evidence that \limdd-based simulation is more efficient than this union.
\end{comment}

\vspace{-1.5em}
Efficient methods for the representation of relevant quantum states and quantum operations are crucial for the simulation and optimization of quantum circuits.
Decision diagrams (DDs), a well-studied data structure originally used to represent Boolean functions, have proven capable of capturing interesting aspects of quantum systems, but their limits are not well understood. 
    In this work, we investigate and bridge the gap between existing DD-based structures and the stabilizer formalism, a well-studied method for simulating quantum circuits in the tractable regime.
%	Decision diagrams (DDs) can represent quantum states and circuits, and have proven a powerful method for simulation and synthesis.
%   We first show that decision diagrams require exponential size to describe certain stabilizer states, despite their reputation for concise representation of structured data.
     We first show that although DDs were suggested to succinctly represent important quantum states,
     they actually require exponential space for a subset of stabilizer states.
	To remedy this, we introduce a more powerful decision diagram variant, called Local Invertible Map-DD (\limdd).
	We prove that the set of quantum states represented by poly-sized \limdds strictly contains the union of stabilizer states and other decision diagram variants.
    We also provide evidence that \limdd-based simulation is capable of efficiently simulating some circuits for which both stabilizer-based and other DD-based methods require exponential time.
    % \todo{Tim: I rewrote after Vedran's comment, please check (Vedran: ``phrasing... do you mean that they can efficiently simulate more than the union of circuits that stab-based and conventional dd-based methods can efficiently simulate?''}
	By uniting two successful approaches, \limdds thus pave the way for fundamentally more powerful solutions for simulation and analysis of quantum computing.
\end{abstract}

