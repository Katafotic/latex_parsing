\section{Preliminaries: decision diagrams and stabilizer states \label{sec:preliminaries}}
%\todo[inline]{mention terminology `low' and `high' for 0-edge and 1-edge}

%Here, we briefly introduce two methods to manipulate and succinctly represent quantum states: decision diagrams and stabilizer states.
%For an introduction to quantum computation, see appendix~\ref{sec:quantum-nutshell}.

The computational unit of quantum computers are quantum bits or qubits.
A single-qubit state is a complex vector $(\alpha_0, \alpha_1)^{T} \in \mathbb{C}^2$ with norm 1, usually written in Dirac notation as $\alpha_0 \ket{0} + \alpha_1 \ket{1}$.
Two quantum states which differ only by a complex multiple are considered equal.
The joint state $\ket{\phi}$ of $n$ quantum bits can be written as 
\begin{equation}
    \label{eq:quantum-state-expansion}
    \sum_{x_1, x_2, \dots, x_n \in \set{0,1}} f(x_1, x_2, \dots, x_n) \ket{x_1}\otimes\ket{x_2}\otimes \dots \otimes\ket{x_n}\vspace{-1em}
\end{equation}
for a function $f: \{0, 1\}^n \rightarrow \mathbb{C}$, where $\otimes$ denotes the tensor product.
An example two-qubit state is $\left( \ket{0} \otimes \ket{0} + i \ket{1} \otimes \ket{1}\right) / \sqrt{2} = \frac{1}{\sqrt{2}} (1,0,0,i)^T$.
Alternatively to \autoref{eq:quantum-state-expansion}, we can recursively describe an $n$-qubit quantum state $\ket{\phi}$ for $n>1$ as
\begin{equation}
\ket{\phi} = \alpha_0 \ket{0} \otimes \ket{\phi_0} + \alpha_1 \ket{1} \otimes \ket{\phi_1}, \text{where $\ket{\phi_0}, \ket{\phi_1}$ are $n$-1-qubit states and $\alpha_0,\alpha_1  \in \mathbb{C}$.}
    \label{eq:quantum-state-recursive}
\end{equation}
An Algebraic Decision Diagram (\add) represents a function of the form $f\colon \{0, 1\}^n \rightarrow \mathbb{C}$, and thus also a quantum state via \autoref{eq:quantum-state-expansion}, see \autoref{fig:qmdd-isoqmdd-exposition} for an example.
%A succint representation of functions of the form $f\colon \{0, 1\}^n \rightarrow \mathbb{C}$, and thus of quantum states via \autoref{eq:quantum-state-expansion}, has been achieved by the use of Algebraic Decision Diagrams (\adds), see \autoref{fig:qmdd-isoqmdd-exposition} for an example.
An \add is a rooted directed acyclic graph (DAG), which has a leaf node for each unique value in the image of $f$, i.e., in $\{f(\vec{x}) \mid \vec{x} \in \{0, 1\}^n \}$.
Each path from the root to a leaf visits nodes representing the variables $x_1, x_2, x_3, x_4$; one variable at each level of the diagram.
The value $f(x_1, \dots, x_n)$ is found by traversing such a path, following the \concept{low edge} (dashed line)
 when $x_i=0$, and the  \concept{high edge} (solid line) when $x_i=1$;
 so, e.g., $f(1,1,1,0) = -i$ in \autoref{fig:qmdd-isoqmdd-exposition}.
Hence every node in an \add, not only the root node, can be said to represent a function.

%\adds for functions $f$ on length-$n$ bitstrings can be defined recursively on $n$: in case $n=1$, then the \add has a single node with two outgoing edge, labelled $0$ and $1$ and pointing to the leaf node holding the value $f(0)$ and $f(1)$ respectively.
%If $f(0) = f(1)$, then the \add only has a single leaf node and both edges point to that node.
%If $n>1$, then and \add for $f$ is constructed by first constructing \adds for the two functions $f_0: (x_2, x_3, \dots, x_n) \mapsto f(0, x_2, x_3, \dots, x_n)$ and $f_1: (x_2, x_3, \dots, x_n) \mapsto f(1, x_2, x_3, \dots, x_n)$.
%An \add for $f_0$ is then made by adding a fresh node and a $0$-edge ($1$-edge) pointing to the root node of the \add of $f_0$ (the \add of $f_1$).
%If the \adds of $f_0$ and $f_1$ are identical, then remove one of them and make both edges point to the remaining one.
%Thus, in an \add, the value $f(x_1, \dots, x_n)$ is found by traversing the path $x_1 \rightarrow x_2 \rightarrow \dots \rightarrow x_n$, e.g. $f(1,1,0) = -1/\sqrt{2}$ in \autoref{fig:add-qmdd-example}.
%Although generally, functions $f: \{0, 1\}^n \rightarrow \mathbb{C}$ have exponentially many partial assignments, many functions encountered in practice have only a polynomial number of unique subfunctions, and therefore can be represented by polynomial-size \add.

A \emph{partial assignment} $(x_1=a_1,\ldots, x_k=a_k)$ to the variables induces a \emph{subfunction} $f_a$, defined as $f_a(x_{k+1},\ldots, x_n)\defn f(a_1,\ldots, a_k,x_{k+1},\ldots, x_n)$.
Two \add nodes representing the same subfunction can be \emph{merged}, i.e., one node is deleted and its incident edges are rerouted to the other.
When all eligible nodes have been merged, an \add is \emph{reduced}.
The nodes of a reduced \add are in one-to-one correspondence with the unique \emph{subfunctions} of $f$.
An \add is a \emph{canonical} representation: a given function has exactly one reduced \add.
\adds can represent both states and matrices, and there are algorithms which multiply a matrix with a vector in \add form.
An \add can represent any quantum state, using exponentially many nodes in the worst case.
A given quantum gate can be compiled into an \add, thus allowing one to simulate any quantum circuit by repeatedly multiplying a gate's matrix with a state. %\todo[inline]{@Lieuwe: David asks for ref here}
%For example, the functions $(x_2, x_2, \dots, x_n) \mapsto f(0, x_2, x_3, \dots, x_n)$ and $(x_4, x_5, \dots, x_n) \mapsto f(0, 1, 1, x_4, x_5, \dots, x_n)$ are subfunctions of $f$.
%At a node in an \add corresponding to the $k$-th variable, the two outgoing arrows correspond to the partial assignments $x_k=0$ and $x_k=1$.

The Quantum Multi-valued Decision Diagram (\qmdd) \cite{miller2006qmdd} improves on the \add representation by also merging two nodes when they represent functions $f,g$ that are related by $f=\lambda\cdot g$ for some $\lambda\in\mathbb C^{\ast}$.
\autoref{fig:qmdd-isoqmdd-exposition} gives an example.
Each edge is labelled with a weight. % (the normalization constant $\lambda$).%$^{\ref{fn:norm}}$
To read a value of $f$, traverse the \qmdd from the root to the leaf just as in an \add, and multiply the weights of all edges on that path.
Like \adds, \qmdds are a canonical representation and can be used to simulate any quantum circuit.

The single-qubit Pauli operators are
$2\times 2$ unitary matrices:
\begin{equation*}
\id[2] \defn \begin{pmatrix} 1 & 0\\ 0 & 1 \end{pmatrix},
X \defn \begin{pmatrix} 0 & 1\\ 1 & 0 \end{pmatrix},
Y \defn \begin{pmatrix} 0 & -i\\ i & 0 \end{pmatrix},
Z \defn \begin{pmatrix} 1 & 0\\ 0 & -1 \end{pmatrix}
\label{eq:pauli-matrices}
\end{equation*}
where $i$ is the complex unit.
An $n$-qubit operator of the form $P_n\otimes\cdots\otimes P_1$ is called a \emph{Pauli string} if $P_j$ are single-qubit Pauli operators.
The $n$-qubit Pauli strings generate a nonabelian group $\Pauli_n$ (under matrix multiplication), consisting of all operators of the form $\lambda P_n\otimes\cdots\otimes P_1$ with $\lambda \in \{\pm 1, \pm i\}$.
Note that the high indices are on the left, in keeping with the custom that the least significant (qu)bit is the first qubit, and the leftmost operator $P_n$ acts on the most significant qubit. 
Pauli operators $A, B$ either commute ($A\cdot B = B\cdot A$) or anticommute ($A \cdot B = - B \cdot A$).

In contrast to decision diagrams, the stabilizer formalism forms a subset of quantum computation that is efficiently simulatable.
A stabilizer state on $n$ qubits can be prepared from the state $\ket{0}^{\otimes n}$ by repeatedly applying any of the following gates (generators of the Clifford set):
\begin{equation}
    H \defn 
        \frac{1}{\sqrt{2}}
        \begin{pmatrix} 1 & 1\\ 1 & -1 \end{pmatrix}
            ,
        S
        \defn
        \begin{pmatrix} 1 & 0\\ 0 & i \end{pmatrix},
            \textnormal{CNOT} \defn 
        \begin{pmatrix}
            1 & 0 & 0 & 0\\
            0 & 1 & 0 & 0\\
            0 & 0 & 0 & 1\\
            0 & 0 & 1 & 0
        \end{pmatrix}.
        \label{eq:clifford-generators}
\end{equation}
There exist $2^{\Theta(n^2)}$ stabilizer states on $n$ qubits \cite{aaronson2008improved}; examples are $\ket{00}$ and $(\ket{00} + \ket{11}) / \sqrt{2}$.
A strict subset of stabilizer states is the set of graph states \cite{hein2006entanglement}.
The relationship between graph states and stabilizer states has been extensively investigated in, e.g., \cite{nest2004graphical,nest2005local}
%\todo{Move to S4.1?}
For an undirected graph $G=(V,E)$, the graph state $\ket{G}$ is the following state on $n$ qubits, where $Z_{jk}$ denotes the controlled $Z$-gate between qubits $j$ and $k$.
\begin{equation}
    \frac{1}{\sqrt{2}^n} \prod_{(j, k) \in E} Z_{jk} \left(\ket{0} + \ket{1}\right)^{\otimes n}, \text{ where e.g. }
        Z_{1,2} = 
        \begin{pmatrix}
            1 & 0 & 0 & 0\\
            0 & 1 & 0 & 0\\
            0 & 0 & 1 & 0\\
            0 & 0 & 0 & -1
        \end{pmatrix} \text{ for } n=2.
	\label{eq:graph-state-definition}
\end{equation}
%Although not every stabilizer state is a graph state, any stabilizer state can be converted into one by local Cliffords (LC)~\cite{nest2004graphical}.
%Moreover, since LC-equivalence between two graphs is efficient~\cite{nest2005local}, so is LC-equivalence between stabilizer states.\todo{Tim: Vedran mentioned this and I added it here, but I am not so sure this is the right place to mention this... All Paulis are CLiffords so checking local-Pauli equivalence is easy for stabilizer states, but we consider arbitrary states...}
An $n$-qubit stabilizer state $\ket{\phi}$ is uniquely specified by the set $S$ of Pauli operators $A \in \Pauli_n$ for which $A\ket{\phi} = \ket{\phi}$.
This set $S$ is an abelian group of $2^n$ elements, succinctly represented by $n$ independent generators.
Since each Pauli generator takes $\mathcal{O}(n)$ space to represent, an $n$-qubit stabilizer is represented by $\mathcal O(n^2)$ bits.
Updating the stabilizer generating set after application of one of the gates from eq.~\eqref{eq:clifford-generators} or a single-qubit computational-basis measurement can be done in polynomial time \cite{gottesman1998heisenberg}.
Also, we note that multiplying two $n$-qubit Pauli strings can be done in $O(n)$ time by using the property of the tensor product $\otimes$ that $(a\otimes b) \cdot (c\otimes d) = (a\cdot c) \otimes (b\cdot d)$.

Stabilizer-rank based methods~\cite{bravyi2016trading,bravyi2017improved,bravyi2019simulation, huang2019approximate,kocia2018stationary,kocia2020improved} extend this approach to families of Clifford circuits with arbitrary input states $\ket{\phi_n}$, enabling the simulation of universal quantum computation in general~\cite{bravyi2005universal}.
By decomposing $\ket{\phi_n}$ as linear combination of $\chi$ stabilizer states, the measurement outcome probabilities can be computed in time $\oh(\chi \cdot \textnormal{poly}(n))$, where the least $\chi$ is referred to as the \concept{stabilizer rank}.
Therefore, stabilizer-rank based methods are efficient for a family of input states $\ket{\phi_n}$ with a stabilizer rank polynomially growing~in~$n$.

In this work we will also consider stabilizer groups of states which are not stabilizer states.
In general, we will refer to an abelian subgroup of $\Pauli_n$, not containing $-\id[2]^{\otimes n}$, as an $n$-qubit \emph{stabilizer subgroup}, which generally has $\leq n$ generators.
Such objects are also studied in the context of simulating mixed states \cite{audenaert2005entanglement} and quantum error correction~\cite{gottesman1997stabilizer}.
Examples of stabilizer subgroups are $\{\id[2]\}$ for $\ket{0} + e^{i\pi/4}\ket{1}$, $\langle -Z\rangle$ for $\ket{1}$ and $\langle X \otimes X\rangle$ for $(\ket{00} + \ket{11}) + 2(\ket{01} + \ket{10})$.

Any $n$-qubit Pauli string can (modulo factor $\in \{\pm 1, \pm i\}$) be written as $(X^{x_n} Z^{z_n}) \otimes \dots \otimes (X^{x_1} Z^{z_1})$ for bits $x_j, z_j, 1 \leq j \leq n$.
We can therefore write an $n$-qubit Pauli string $P$ as length-$2n$ binary vector 
\[
    (\underbrace{x_n, x_{n-1}, \dots x_1}_{\textnormal{X block}} | \underbrace{z_n, z_{n-1}, \dots, z_1}_{\textnormal{Z block}})
    ,
\] 
where we added the horizontal bar ($|$) only to guide the eye.
We will refer to such vectors as \emph{check vectors}.
For example $X \sim (1, 0)$ and $Z \otimes Y \sim (0, 1 | 1, 1)$ \cite{aaronson2008improved}.
A set of $k$ Pauli strings thus can be written as $2n\times k$ binary matrix, often called \emph{check matrix}, e.g.
\[
    \begin{pmatrix}
        X &\otimes& X &\otimes& X\\
        \id[2] &\otimes& Z &\otimes& Y
    \end{pmatrix}
    \sim
    \begin{pmatrix}
        1& 1& 1& |& 0& 0& 0\\
        0& 0& 1& |& 0& 1& 1
    \end{pmatrix}
    .
\]
This equivalence induces an ordering on Pauli strings following the lexicographic ordering on bit strings. %, defined as $\vec{y} \leq \vec{z}$ if either $\vec{y} = \vec{z}$ or else if $\vec{y}$ $\vec{z}$ agree on the first $k$ elements but $\vec{y}_{k+1} < \vec{z}_{k+1}$, where $0<1$.
For example, $X<Y$ because $(1|0) < (1|1)$ and $Z\otimes \id[2] < Z \otimes X$ because $(0 0 | 1 0) < (0 1 | 1 0)$.
Furthermore, if $P, Q$ are Pauli strings corresponding to binary vectors $\vec{x}^P, \vec{z}^P$ and $\vec{x}^Q, \vec{z}^Q$, then 
\[
P \cdot Q \propto
\bigotimes_{j=1}^n
\left(X^{x^P_j} Z^{z^P_j}\right) \left(X^{x^Q_j} Z^{z^Q_j}\right) 
=
\bigotimes_{j=1}^n
\left(X^{x^P_j \oplus x^Q_j } Z^{z^P_j \oplus z^Q_j} \right)
\]
and therefore the group of $n$-qubit Pauli strings with multiplication (disregarding factors) is group isomorphic to the vector space $\{0, 1\}^{2n}$ with bitwise addition (i.e., exclusive or; `xor').
Consequently, many efficient algorithms for linear-algebra problems carry over to sets of Pauli strings.
In particular, if $G = \{g_1, \dots, g_k\}$ are length$-2n$ binary vectors (/ $n$-qubit Pauli strings) with $k\leq n$, then we can efficiently perform the following operations.
\begin{description}
%    \item[\emph{Orthogonalization:}] convert $G$ to a (potentially smaller) independent set, using the Gram-Schmidt procedure, in $O(n^3)$ time.
    \item[\emph{RREF:}] bring $G$ into a reduced-row echelon form (RREF) using Gauss-Jordan elimination (both standard linear algebra notions) where each row (in check matrix form) has strictly more leading zeroes than the row above.
        The RREF is achievable by $O(k^2)$ row additions (/~multiplications modulo factor) and thus $O(k^2 \cdot n)$ time (see \cite{berg2020circuit} for a similar algorithm).
        In the RREF, the first $1$ after the leading zeroes in a row is called a `pivot'.
    \item[\emph{Independent Set}] convert $G$ to a (potentially smaller) independent set by performing the RREF procedure and discarding resulting all-zero rows.
    \item[\emph{Membership:}] determining whether a given a vector (/~Pauli string) $h$ has a decomposition in elements of $G$.
        This task can be reduced to independence by first getting $G^{\textnormal{RREF}}$ by applying RREF, followed by adding $h$ to $G$ and performing the Independent-Set procedure. The result has $|G^{\textnormal{RREF}}|$ rows if $h\in \langle G\rangle$, and $|G^{\textnormal{RREF}}| + 1$ rows otherwise.
    \item[\emph{Intersection:}] determine all Pauli strings which, modulo a factor, are contained in both $G_A$ and $G_B$, where $G_A, G_B$ are generator sets for $n$-qubit stabilizer subgroups.
        This can be achieved using the Zassenhaus algorithm \cite{LUKS1997335} in time $O(n^3)$.
    \item[\emph{Division remainder:}] given a vector $h$  (/~Pauli string $h$), determine \mbox{$ h^{\textnormal{rem}} := \min_{g\in \langle G\rangle} \{ g  h\}$} (minimum in the lexicographic ordering) where $\oplus$ denotes bitwise XOR (/~factor-discarding multiplication).
        We do so in the check matrix picture by bringing $G$ into RREF, and then making the check vector of $h$ contain as many zeroes as possible by adding rows from $G$:
	\begin{algorithmic}[1]
        \For{column index $j=1$ to $2n$}
        \If{$h_j = 1$ and $G$ has a row $g_i$ with its pivot at position $j$}
		$h := h \oplus g_i$
        \EndIf
        \EndFor
	\end{algorithmic}
        The resulting  $h$ is $h^{\textnormal{rem}}$.
This algorithm's runtime is dominated by the RREF step; $O(n^3)$.
        %\todo{It is hard to follow. An example might help. Perhaps indeed explain where you need it?}
%        \todo[inline]{Tim: this algorithm is obvious but I have not been able to find a reference.... Need to move to elsewhere?}
% AL: I like this overview. If it is trivial let's leave it here. People might still refer to it, and we avoid claiming trivial stuff.
\end{description}

In this work, we will consider the group of $n$-qubit Pauli operators $\lambda P_n \otimes \dots P_n$ for arbitrary $\lambda \in \mathbb{C}-\{0\}$, denoted as $\paulilim_n$.
Since each stabilizer $\lambda P \in \paulilim_n$ has factor $\lambda =\pm 1$ (follows from
$ (\lambda P)\ket{\phi} = (\lambda P)^2 \ket{\phi} =  \lambda^2 \id \ket{\phi}  =\ket{\phi}$, hence $\lambda^2 = 1$), the stabilizer subgroups in $\paulilim_n$ are the same as in $\pauli_n$.
As extension of the check matrix form to $A \in \paulilim_n$, we write $A = r \cdot e^{i\theta} \cdot P_n \otimes ...\otimes P_1$, for $r \in \mathbb{R}_{> 0}$ and $\theta\in [0, 2\pi)$ and represent $A$ by a length-$(2n+2)$ vector where the last entries store $r$ and $\theta$, e.g.:
\[
    \begin{pmatrix}
        3X &\otimes& X &\otimes& X\\
        -\frac{1}{2} i\id[2] &\otimes& Z &\otimes& Y
    \end{pmatrix}
    \sim
    \begin{pmatrix}
        1& 1& 1& |& 0& 0& 0& |&3 & 0\\
        0& 0& 1& |& 0& 1& 1& |&\frac{1}{2} & \frac{3\pi}{2}
    \end{pmatrix}
\]
where we used $3 = 3\cdot e^{i\cdot 0}$ and $-\frac{1}{2}i = \frac{1}{2} \cdot e^{3\pi i/2}$.
The ordering on real numbers induces a lexicographic ordering (from left to right) on such extended check vectors, for example $(1, 1, | 0, 0 | 3, \frac{1}{2} ) < (1, 1 | 1, 0 | 2, 0)$.
Let us stress that the factor encoding $(r, \theta)$ is less significant than the Pauli string encoding $(x_n, \dots, x_1 | z_n, \dots, z_1)$.
As a consequence, we can greedily determine the minimum of two Pauli operators.
%We state this remark separately because we will explicitly use it for proving correctness of our algorithms in~\autoref{sec:choose-canonical-isomorphism-pauli}--\ref{sec:pauli-isomorphism-detection}.

%\def\rp{r}
%\def\thetap{\theta}
%\def\rq{r'}
%\def\thetaq{\theta'}
%\begin{remark}
%    \label{remark:ordering}
%    Checking which of two Pauli operators $\rp e^{i\thetap} P, \rq e^{i\thetaq} P' \in \paulilim_n$ is smaller can be performed in two greedy steps: first, declare $\rp e^{i\thetap} P <  \rq e^{i\thetaq} P'$ if $P < P'$.
%    Otherwise (i.e., if $P = P'$), then proceed to comparing the factors and declare $\rp e^{i\thetap} P <  \rq e^{i\thetaq} P'$ if and only if $(\rp, \thetap) < (\rq, \thetaq)$.
%\end{remark}

Finally, we emphasize that the algorithms above rely on row addition, which is a commutative operation.
%are correct because Pauli operators with factor-ignoring multiplication are group isomorphic to binary vectors with xor.
Since conventional (i.e., factor-respecting) multiplication of Pauli operators is not commutative, the algorithms above are not straightforwardly applicable to (nonabelian subgroups of) $\paulilim_n$.
(For abelian subgroups of $\paulilim_n$, such as stabilizer subgroups \cite{aaronson2008improved}, the algorithms still do work.)
Fortunately, since Pauli strings either commute or anti-commute, row addition may only yield an factors up to the $\pm$ sign, not the resulting Pauli strings.
This feature, combined with the stipulated order assigning least significance to the factor,
enables us to invoke the algorithms above as subroutine, with postpocessing to obtain the correct factor.
We will do so in~\autoref{sec:choose-canonical-isomorphism-pauli}--\ref{sec:pauli-isomorphism-detection}.
%
%
%, which may yield incorrect output (for example, the intersection of $\langle Z \rangle$ and $\langle -Z \rangle$ is $\{\id[2]\}$, while if we discard scalars, the Intersection algorithm above would also return $Z$ as element in the intersection).
%However, because $\paulilim_n$ with factor-respecting multiplication is a non-abelian group while row addition is, there is no commutative row addition operation that preserves factors in the Pauli picture.










