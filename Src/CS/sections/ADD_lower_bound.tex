\subsection{Exponential separation of QMDD from ADD}

Algebraic Decision Diagrams (\textsf{ADD}) are a precursor to \textsf{QMDD}s.
The \textsf{QMDD} data structure groups together nodes representing subfunctions which differ only by a multiplicative constant.
By labelling the edges with appropriate weights, no information about the state is lost.
In this Section, we show that this leads to an exponential decrease in the size of the diagram.
Put another way, we have $\textsf{ADD}\subsetneq \textsf{QMDD}$.
In the literature, \textsf{ADD} is sometimes called \textsf{QuiDD}.

\begin{theorem}
	There is an infinite family of quantum states $\{\ket{\phi_n}_n\}_{n}$ such that every reduced, ordered \textsf{ADD} needs $\Theta(2^n)$ nodes to store $\ket{\phi_n}$, but every \textsf{QMDD} needs only $\Theta(n)$ nodes.
\end{theorem}
\begin{proof}
	The state is the Fourier-transformed state
	\begin{align}
		\ket{\phi_n}=(\ket{0}+e^{i\pi})\otimes (\ket{0}+e^{i\pi2^{-1}})\otimes \cdots \otimes (\ket{0}+e^{i\pi2^{-n+1}})
	\end{align}
	Since the state is a product state, its QMDD only has $n$ nodes, not counting the \textsf{leaf} node.
	
	However, the amplitudes of each basis vector is different, so the \textsf{ADD} has an exponential number of leaves, each containing a different amplitude. Namely, the amplitude in $\ket{\phi}_n$ of basis vector $\ket{x}_n$ is $\frac{1}{\sqrt{2^n}}e^{i\pi x2^{-n}}$.
\end{proof}